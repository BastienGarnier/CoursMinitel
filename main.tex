\documentclass{minitelreport}

\graphicspath{{sections/part1/images/}{sections/part2/images/}{sections/part3/images/}}

\usepackage[backend=biber,style=numeric, sorting=nty, defernumbers=true]{biblatex}
\addbibresource{references/bibliographie.bib}
\addbibresource{references/webographie.bib}
\addglobalbib{./references/bibliographie.bib}
\addglobalbib{./references/webographie.bib}

%%%%%%%%%%%%
% Counters %
%%%%%%%%%%%%

% Compteur des exercices

\newcounter{exercisescounter}
\setcounter{exercisescounter}{0}

\newcommand{\exercise}[1]{
	\refstepcounter{exercisescounter}
	\textbf{Exercice \arabic{exercisescounter}\IfNoValueF{#1}{ (#1)}. \label{exercise:#1}}
}

\newcommand{\refexercise}[1]{
	\textbf{Exercice \ref{exercise:#1}.}
}

% Compteur de solutions

\newcounter{solutioncounter}
\setcounter{solutioncounter}{0}

\newcommand{\solution}[1]{
	\stepcounter{solutioncounter}
	\textbf{Solution \arabic{solutioncounter}\IfNoValueF{#1}{ (#1)}.}
}

\newcommand{\listdefinitionsname}{Liste des définitions}
\newlistof{definition}{def}{\listdefinitionsname}

\newcommand{\definition}[2]{
	\refstepcounter{definition}
	\begin{minitelbasicbox}{\textbf{Définition \arabic{definition}\IfNoValueF{#1}{ : #1}}}
#2
	\end{minitelbasicbox}
	\addcontentsline{def}{figure}{\protect\numberline{\thesection.\thedefinition}\qquad#1}
}

\newcounter{propositioncounter}
\setcounter{propositioncounter}{0}

\newcommand{\proposition}[1]{
	\stepcounter{propositioncounter}
	\textbf{Proposition \arabic{propositioncounter}\ifthenelse{\equal{#1}{}}{}{ (#1)}. }
}

\usepackage{subfiles}
\begin{document}
\title{Informatique}
\subtitle{Une introduction à la pratique}
\variant{je-sais-pas-combien-mais-pas-encore-la-1.0}
\author{A. Nonyme}
\maketitle
\begin{refsection}
\section*{Préface}
\addcontentsline{toc}{section}{Préface}
\subfile{sections/preface}
\tableofcontents
\listoftables
\listofdefinition
\newpage
\part{Prolégomènes}
	\phantomsection
	\section*{Introduction de l'introduction}
	\addcontentsline{toc}{section}{Introduction de l'introduction}
	\subfile{sections/part1/introduction}
	\chapter{Présentation du sujet}
		\subfile{sections/part1/chapter1/chapter1}
	\chapter{Programmer un ordinateur}
		\subfile{sections/part1/chapter2/chapter2}
\part{Du langage C}
	\phantomsection
	\section*{Introduction}
	\addcontentsline{toc}{section}{Introduction}
	\subfile{sections/part2/introduction}
	\chapter{Fondamentaux du langage}
		\subfile{sections/part2/chapter1/chapter1}
	\chapter{Bases du langage}
		\section{Variables}
		\subfile{sections/part2/chapter2/variables}
		\section{Formatage de texte}
		\subfile{sections/part2/chapter2/formatage}
		\section{Opérateurs sur les variables}
		\subfile{sections/part2/chapter2/operateurs}
		\section{Projection de type}
		\subfile{sections/part2/chapter2/projections}
		\section{Structures de contrôle}
		\subfile{sections/part2/chapter2/controle_structures}
		\section{Routines}
		\subfile{sections/part2/chapter2/routines}
		\section{Pointeurs}
		\subfile{sections/part2/chapter2/pointeurs}
		\section{Interagir avec les flux standards}
		\subfile{sections/part2/chapter2/stdstreams}
		\section{Tableaux statiques}
		\subfile{sections/part2/chapter2/tab_statiques}
		\section{Tableaux dynamiques}
		\subfile{sections/part2/chapter2/tab_dynamiques}
		\section{Tableaux multidimensionnels}
		\subfile{sections/part2/chapter2/tab_multidim}
		\section{Structures}
		\subfile{sections/part2/chapter2/structures}
		\section{Modulation et entêtes}
		\subfile{sections/part2/chapter2/modulation}
		\section{Chaînes de caractères}
		\subfile{sections/part2/chapter2/chaines_caracteres}
		\section{Les flux de fichiers}
		\subfile{sections/part2/chapter2/flux_fichiers}
	\chapter{Concepts avancés (en cours de rédaction)}
		\section{Paramètres d'un programme}
		\subfile{sections/part2/chapter3/prgm_args}
		\section{Unions}
		\subfile{sections/part2/chapter3/union}
		\section{Champs de bits}
		\subfile{sections/part2/chapter3/champs_bits}
		\section{Classes de stockage}
		\subfile{sections/part2/chapter3/classes_stockage}
		\section{Espaces locaux et espace global}
		\subfile{sections/part2/chapter3/volatile}
		\section{Construction de littéraux}
		\subfile{sections/part2/chapter3/litteraux}
		\section{Pointeurs de routines}
		\subfile{sections/part2/chapter3/pointeurs_routines}
		\section{Tableaux de routines}
		\subfile{sections/part2/chapter3/tableaux_routines}
		\section{Fonctions variadiques}
		\subfile{sections/part2/chapter3/variadic}
		\section{Alignement}
		\subfile{sections/part2/chapter3/alignement}
		\section{Directives de préprocesseurs (2)}
		\subfile{sections/part2/chapter3/preprocessor}
		\section{Assembleur en C}
		\subfile{sections/part2/chapter3/assembly}
		\section{La virgule}
		\subfile{sections/part2/chapter3/comma}
		\section{Tricks récréatifs (ft. Basile)}
		\subfile{sections/part2/chapter3/tricks}
\part{Le vrai monde de la réalité réelle}
	\phantomsection
	\section*{Introduction}
	\addcontentsline{toc}{section}{Introduction}
	\subfile{sections/part3/introduction}
	\chapter{Outils utiles à la programmation}
		\section{La documentation}
			\subsection{Liste des sites d'information}
				\subsubsection{\textit{Wikipedia.org}}	
				\subsubsection{\textit{man7.org}}
				\subsubsection{Microsoft Doc}
				\subsubsection{Zeste de Savoir}
				\subsubsection{\textit{os-dev.org}}
				\subsubsection{\textit{developpez.com}}
			\subsection{Documentation hors-ligne}
				\subsubsection{La commande \textit{man} sous Linux}
				La commande \textit{man} du terminal Linux est probablement la commande la plus utile au développeur C.
			\subsection{Lire un prototype}
			Abbréviations classiques : fd, buffer, etc...
			\subsection{Liste des sites de forum}
				\subsubsection{Stack Overflow}
				\subsubsection{developpez.com}
			\subsection{À propos des GPTs}
			\subfile{sections/part3/chapter1/gpt}
		\section{Makefiles}
		\subfile{sections/part3/chapter1/makefile}
		\section{Créer sa bibliothèque externe}
		\section{Utiliser un débogueur}
		Le déboguage est une composante fondamentale du développement logiciel. En effet, il est fondamentale de pouvoir identifier les fautes dans un code, en particulier si celui-ci rencontre des erreurs fatales à son fonctionnement (i.e. \textit{crash}). \textit{GDB} (\textit{\underline{G}NU \underline{D}e\underline{B}ugger}) est un logiciel qui permet d'opérer le déboguage de programmes complexes. Seront détaillés dans la suite les fonctionnalités basiques qu'il propose.
	\chapter{Méthodes de programmation propre}
		\section{La gestion des erreurs}
		\subfile{sections/part3/chapter2/erreurs}
		\section{Tests unitaires}
		\subfile{sections/part3/chapter2/unittests}
		\section{À propos du style}
		\subfile{sections/part3/chapter2/style}
\part{Annexes}
\chapter{Correction des \theexercisescounter\ exercices}
	\subfile{sections/solutions}
\phantomsection
\printbibheading
\printbibliography[type=book,heading=subbibliography,title={Livres}]
\printbibliography[type=manual, heading=subbibliography, title={Manuels et documentation}]
\printbibliography[type=online, heading=subbibliography, title={Autres liens}]
\addcontentsline{toc}{chapter}{Bibliographie}
\end{refsection}
\end{document}