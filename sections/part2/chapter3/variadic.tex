\documentclass[../../../main.tex]{subfiles}
\begin{document}
\subsection{Motivation}
Comprendre comment la fonction \textsf{printf} du C permet de prendre un nombre d'arguments variable à son appel, parce-que j'espère que tous les lecteurs arrivés à ce point se sont posés la question\dots Ah ! Et faire pareil aussi, ce serait chouette $\wedge\wedge$
\subsection{Appels de routines}
Il existe plusieurs standards d'appel de routines. Lors d'un appel de routine \textit{suivant le standard du langage C}\footnote{Pour des raisons d'optimisation, le compilateur peut utiliser d'autres standards, voir \url{https://en.wikipedia.org/wiki/X86_calling_conventions} pour des détails.}, les arguments sont poussés sur la pile suivant le schéma simplifié suivant :

\begin{minipage}{\textwidth}
	\begin{center}
		\hspace{2.5cm}\includesvg[width=.5\textwidth]{routine_call}
	\end{center}
\end{minipage} 

Ils sont ensuite dépilés lors de la sortie de la routine, pour revenir à l'environnement précédent l'appel.

Le compilateur doit donc savoir à la compilation combien d'arguments sont présents lors de l'appel pour pouvoir les empiler. De même pour pouvoir y accéder à l'intérieur d'une routine. Le nombre d'arguments ne peut donc être dynamique et déterminé par une variable du programme. Il faut qu'à l'appel de la routine, le nombre d'arguments soit déterminé. Il faut également que la taille en octets de chacun des arguments puisse être déterminé à la fois à la compilation et à la lecture de ceux-ci dans la routine. \\
C'est le cas des appels à \textsf{printf} :
\begin{minted}[linenos=false]{c}
// nombre et taille des arguments connues à la compilation
// la première chaîne de caractères donne ces informations à la routine
printf("%d + %d = %d\n", a, b, a + b); 
\end{minted}
Le nombre d'arguments à fournir n'est pas donné par la routine intrinsèquement. Il est donné par le programmeur, qui doit tout de même se conformer à la spécification de la routine.
\subsection{Syntaxe et module \textit{stdarg}}
Une routine variadique doit contenir \textit{a minima} un paramètre nommé qui permette de déterminer le début des paramètres variadiques.

Pour définir une routine variadique, on utilise la syntaxe suivante :
\begin{minted}[linenos=false]{c}
TYPE function(des paramètres déterminés, ...);
\end{minted}
Les trois points \dots ne sont pas là comme dans le reste du livre pour indiquer ``mettre ce que l'on veut''. Il s'agit véritablement de la syntaxe permettant de définir des routines variadiques. Ainsi, on peut donner le prototype d'une fonction calculant la somme d'un nombre $n > 0$ d'entiers :
\begin{minted}[linenos=false]{c}
int variadic_sum(unsigned int n, ...);
\end{minted}
qui peut par exemple être appelée par :
\begin{minted}[linenos=false]{c}
int x = variadic_sum(3, 7, 9, -5); // x = 7 + 9 - 5
\end{minted}
Intéressons nous maintenant à l'implantation d'une telle fonction \textsf{variadic\_sum}.

Il est en théorie possible, en utilisant l'adresse du premier argument, de retrouver les arguments donnés à la routine. Il y a certains problèmes à cela :
\begin{itemize}
	\item les conventions d'appel de routines et de passage d'arguments dépendent à la fois du processeur et du compilateur
	\item si on active les optimisations du compilateur, les conventions d'appel changent
\end{itemize} 
Ainsi, le seul qui possède toutes les informations pour lire de manière certaine les arguments passés à la routine variadique est \textit{le compilateur}. 

Heureusement, celui-ci possède des routines intégrées capables d'initialiser et d'effectuer la lecture des arguments variadiques. Le module \textit{stdarg} de la bibliothèque standard permet d'appeler ces routines intégrées, dites \textit{``builtins''}\footnote{\url{https://github.com/gcc-mirror/gcc/blob/master/gcc/ginclude/stdarg.h} pour le vérifier.}.

Il définit pour cela le type \textsf{va\_list} qui permet de parcourir les arguments variadiques suivant un paramètre indiqué. Il s'agit en fait d'un pointeur vers les arguments. On utilise pour initialiser et fermer proprement la lecture des arguments de deux macros :
\begin{itemize}
	\item \textsf{va\_start(va\_list args, PARAMETER\_NAME)}
	\item \textsf{va\_end(va\_list args)}
\end{itemize}
\begin{minted}[linenos=false]{c}
#include <stdarg.h>

int variadic_sum(unsigned int n, ...) {
	va_list args;
	va_start(args, n); // on démarre la lecture des arguments après le paramètre 'n'

	// lecture/utilisation des arguments variadiques

	va_end(args);
}
\end{minted}
On dispose de deux macros pour lire les arguments :
\begin{itemize}
	\item \textsf{va\_arg(va\_list args, TYPE)} : renvoie l'argument suivant, de type \textsf{TYPE}, et passe à l'argument suivant. 
	\item \textsf{va\_copy(va\_list dest, va\_list source)} : initalise une liste d'arguments \textit{dest} comme une copie de la liste \textit{source}
\end{itemize}
Ainsi, on peut écrire :
\begin{minted}[linenos=false]{c}
#include <stdarg.h>

int variadic_sum(unsigned int n, ...) {
	int sum = 0;
	va_list args;
	va_start(args, n); // on démarre la lecture des arguments après le paramètre 'n'
	for (unsigned int i = 0; i < n; i++) {
		sum += va_arg(args, int);
	}
	va_end(args);
	return sum;
}
\end{minted}
\subsection{Macros variadiques}
Les macros fonctionnelles du préprocesseur peuvent aussi être variadiques et prendre un nombre quelconque de paramètre. La syntaxe et les possibilités sont beaucoup plus limités que dans le cas des routines variadiques ``classiques''.

La syntaxe est la suivante :
\begin{minted}{c}
#define VAR(named_args, ...) // code using __VA_ARGS__ symbol
\end{minted}
En fait, moins qu'une macro à nombre indéfini de variables, il s'agit d'une possibilité de passer à la macro un argument qui contient des virgules.

L'argument ``\dots'' peut contenir des virgules et est nommé dans la macro par \textsf{\_\_VA\_ARGS\_\_}.

Ainsi, on peut écrire :
\begin{minted}{c}
#define debug_printf(...) \
	fprintf(stderr, "At line %d : ", __LINE__); \
	fprintf(stderr, __VA_ARGS__); \
	fprintf(stderr, "\n")
// plus loin
debug_printf("Dividing by %d", x);
\end{minted}
\subsection{Exercices}
\exercise{À un doigt du zéro}{10} Écrire une fonction \textsf{unsigned int mult(unsigned int first, \dots);} qui parcourt ses arguments jusqu'au premier 0 rencontré, et renvoie le produit de ces arguments, 0 exclu.

\exercise{Fonctions à paramètres par défaut}{20}\\
On veut implanter en C le concept de paramètres par défaut. Il s'agit de définir une fonction dont certains paramètres, nommés, ont une valeur par défaut à l'appel et n'ont pas besoin d'être précisés. Par exemple, on peut imaginer une fonction d'addition avec retenue avec un paramètre $r$ dont la valeur par défaut est $0$. Idéalement, on voudrait pouvoir écrire :
\begin{minted}{c}
add_r(4, 5); // par défaut, retenue = 0 -> resultat = 4 + 5
add_r(4, 5, .r = 1); // on modifie la valeur par défaut à 1 -> resultat = 4 + 5 + 1
\end{minted}
\begin{enumerate}
	\item Écrire une structure \mintinline{c}{add_r_opt} contenant un unique champ $r$
	\item Écrire une fonction \mintinline{c}{int add_r_(int a, int b, add_r_opt opt);} dont le comportement est décrit ci-dessus.
	\item Écrire une macro \mintinline{c}{add_r} qui se comporte exactement comme ci-dessus.
	\item Faire de même pour la fonction \mintinline{c}{int atoi(char* string)} de l'\refexercise{AtoI} en ajoutant un paramètre $base$ qui correspond à la base dans laquelle est écrit le nombre à convertir ($base = 10$ par défaut).
\end{enumerate}
\end{document}