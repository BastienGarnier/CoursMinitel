\documentclass[../../../main.tex]{subfiles}
\graphicspath{{sections/part2/images/}}
\begin{document}
Ceci n’est pas une blague. Il s’agit d’une section traitant de la virgule ``,'' en C, à la fois comme
opérateur entre différentes expressions et comme séparateur de déclarations.

Cette section n’apparaît qu’à la fin du cours de C au vu de l’anecdoticité\footnote{Je ne sais pas si le mot existe\dots voir même que je suis certain qu'il n'existe pas *\textbackslash{}\_°! °\_/*} de cette caractéristique du
langage. Il s’agit uniquement d’être exhaustif.

Et très honnêtement, ça ne sert pas à grand-chose\dots Mais si jamais, c’est là !
\subsection{La virgule comme opérateur}
La virgule sert à effectuer des opérations à effets de bord lors de l’écriture d’expressions.

\textbf{Rappel :} Une routine à effet de bord est une routine qui agit autrement que par sa valeur de retour,
c’est-à-dire qui agit en dehors de son environnement local. Par exemple, une routine qui affiche du
texte est une routine à effet de bord. Une routine qui modifie la valeur d’un tableau entré en argument
est une routine à effet de bord.

La syntaxe est la suivante :
\begin{minted}[linenos=false]{c}
expression = (expr1, ..., exprn);
\end{minted}
Elle est équivalente à :
\begin{minted}[linenos=false]{c}
expr1;
expr2;
...;
expr<n-1>;
expression = exprn;
\end{minted}
Un exemple purement technique consistant à compter le nombre d’assignations à des variables :*
\begin{minted}{c}
#include <stdio.h>
#include <stdio.h>

#define ASSIGN(X) (i++, X)

int main(int argc, char **argv) {
	unsigned int i = 0;
	int a = ASSIGN(42);
	int b = ASSIGN(64);
	b = ASSIGN(a + b);
	int c = ASSIGN(i*b + a);
	printf("Nombre d'assignations : %u\n", i);
	return EXIT_SUCCESS;
}
\end{minted}
Un second exemple à peine moins inutile pour l’échange de valeur de deux variables :
\begin{minted}[linenos=false]{c}
int a = 42;
int b = 64;
int tmp = (tmp = a, a = b, b = tmp);
print("a = %d et b = %d\n", a, b);
\end{minted}
\textbf{Remarque :} Oublier les parenthèses dans ce deuxième exemple est rédhibitoire puisque le compilateur
va considérer que les variables a et b sont redéfinies dans le même bloc, ce qui amène à une erreur de
compilation.

En particulier, il faut rester attentif à la priorité des opérateurs. L’opérateur d’assignation est en effet
prioritaire sur la virgule.
\begin{minted}[linenos=false]{c}
int x = 1, 2, 3;
\end{minted}
est incorrect car équivalent à :
\begin{minted}[linenos=false]{c}
(((int x = 1), 2), 3);
\end{minted}
qui n’a pas de sens.
\subsection{La virgule comme séparateur}
La virgule sert également dans un cas déjà observé : la séparation des déclarations de variables.
\begin{minted}[linenos=false]{c}
int a = 3, b = a + 1;
\end{minted}
\subsection{Conclusion}
Ben voilà c’est tout...
\end{document}