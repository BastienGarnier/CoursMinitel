\providecommand{\topdir}{../..}
\documentclass[../../main.tex]{subfiles}
\graphicspath{{sections/part2/images/}}
\begin{document}
Cette deuxième partie du cours d'informatique porte sur la programmation en langage C, pour les raisons susdites dans la partie précédente.
 
L'objectif est d'offrir une vue qui se veut assez complète mais surtout rigoureuse\footnote{Rigueur entendue comme non nécessairement formelle, c'est-à-dire qu'on n'explicitera pas la théorie du langage et la théorie de la compilation sous-jacentes}, afin d'amener le lecteur à une maîtrise du langage suffisante pour l'écriture de programmes quelques sur un ordinateur personnel. Le cours ne cherche donc pas à se spécialiser dans un domaine quelconque et aborde de manière générale la programmation en C\footnote{En particulier, rien ne sera dit vis-à-vis des spécificités de la programmation de systèmes embarquées bien qu'il s'agisse d'une spécialité de la formation ISMIN.}.
 
Dans cette optique, la progression se veut tout à fait linéaire et progressive dans le sens où chaque concept présenté ne nécessite pour sa bonne compréhension que les concepts présentés précédemment.
 
La structure générale du texte est thématique et chaque thème est suivi d'exercices pour assurer l'assimilation des notions abordées. On distingue quatre classes de thèmes :
\begin{itemize}
	\item les fondamentaux relatifs à la syntaxe très générale du langage C et à quelques points généraux nécessaires à une vue d'ensemble
	\item les bases qui présentent l'essentiel du langage C, c'est-à-dire ce qui suffit à écrire des programmes quelconques en langage C
	\item les concepts avancées qui abordent certaines subtilités et particularités de première nécessité douteuse mais d'utilité avéré
	\item des concepts divers qui ne concernent pas le langage C en tant que tel mais plutôt certains outils et certaines méthodes nécessaires pour une programmation stable éclaircie du plus gros des embûches menaçant le débutant 
\end{itemize}
\textbf{À propos de l'universalité ou de la particularité des langages de programmation et en particulier du C :}
 
Il peut venir à l'esprit à la lecture de ce document deux idées opposées. D'abord que les concepts présentés à propos du langage sont présents dans d'autres langages comme le Python, et semblent par ailleurs présenter un caractère universel à la programmation (on pensera par exemple au concept de \textit{variable}). Par ailleurs, certaines particularités pourraient n'être que spécifiques au C et généralisable à aucun autre langage, comme cela peut être pensé des \textit{classes de stockage}.
 
On remarque d'abord qu'il est absurde penser qu'un langage puisse être tout à fait particulier sans jamais présenter aucun point commun avec quelque langage que ce soit. Cela semble intuitif et est par ailleurs démontré mathématiquement par l'équivalence stricte des langages impératifs découlant des modèles de la machine de Turing et du $\lambda-$calcul.\newline
Nuançons malgré tout. Certains concepts apparaissent dans la \textit{plupart} des langages impératifs car ils corroborent l'intuition algorithmique humaine. Pourtant, ces concepts ne sont absolument pas inhérents aux théories du calcul dont découlent ces langages. Par exemple, la notion de \textit{variable à valeur dynamique} peut être ignoré dans certains langages comme le \textit{Haskell}, dont le principe est très calqué sur le $\lambda$-calcul et se trouve ainsi purement fonctionnel. Par ailleurs, d'autres concepts ne sont pas des conséquences directes de la théorie mais se trouvent être inhérents à la pratique, c'est-à-dire à l'implantation des langages sur un ordinateur. Ainsi, la nécessité de manipuler des adresses mémoires sur un ordinateur classique rend le concept de pointeur presque universel. Si l'utilisateur ne les utilise pas consciemment, les pointeurs apparaissent ainsi systématiquement dans le fonctionnement interne du langage.\footnote{Sauf dans le cas de langages ne permettant que l'utilisation des registres du processeur, mais je n'ai jamais entendu parlé de tels langages}

\hrulefill
\newpage
\end{document}