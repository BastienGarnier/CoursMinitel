\documentclass[../../../main.tex]{subfiles}
\begin{document}
C'est pour cela que les tests unitaires existent. L'objectif des programmes de tests unitaires est de
vérifier des routines isolées (unitaires) dans des cas particulier d'exécution qui pourrait potentiellement
amener à une erreur en comparant la sortie de chaque routine sur une entrée $e$ prédéterminée avec une
sortie attendue $s(e)$. Il faut alors tester tous les entrées les plus particulières.

Chaque routine d'un programme doit en théorie être testée par des tests unitaires. Dans la pratique,
du fait des ressources humaines nécessaires à la réalisation de ces tests, on ne test que les routines
``non triviales'' d'un programme, c'est-à-dire celles qui peuvent présenter des fautes potentielles.

L'importance des tests unitaires est multiple :
\begin{itemize}
	\item Correction efficiente des bogues : une routine bogué est plus facilement corrigeable tôt que lorsque
la moitié du programme en est dépendant.
	\item Documentation : les tests unitaires servent d'exemples sur la manière dont les routines doivent
être utilisées
	\item Confiance en la qualité du code
\end{itemize}
\end{document}