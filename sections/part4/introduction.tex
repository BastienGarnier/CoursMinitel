\providecommand{\topdir}{../..}
\documentclass[../../main.tex]{subfiles}
\graphicspath{{sections/part4/images/}}
\begin{document}
Cette quatrième partie du cours d'informatique prend le contrepied de la troisième partie. Son contenu de fond appartient à un cadre beaucoup plus pratique de la programmation.
 
Si le/la lecteur/trice assidu(e) et consciencieux/se des trois parties précédentes doit maintenant maîtriser assez bien le langage C\footnote{Ainsi que nombre de notions gravitant autour comme les représentations binaires des nombres ou quelques principes de la programmation système} d'un point de vue des connaissances syntaxiques et sémantiques, et doit également posséder suffisamment de connaissances théoriques pour comprendre les fondements de l'informatique, cela ne suffit pas à en faire un(e) développeur/se compétent(e). Il lui manque encore certains points fondamentaux qui agissent plus au niveau professionnel que personnel. La programmation n'est en général pas un travail absolument individuel puisqu'un programme est souvent destiné à d'autres qu'à soi.

Ont été régulièrement donnés quelques indices vis-à-vis du développement de projets de grandes tailles, et sur la nécessité d'une méthodologie d'approche qui permette de rester efficace/efficient sur le long terme, au fur-et-à-mesure que le projet devient trop gros ou grand pour être facilement pensé dans sa globalité par un unique individu.\footnote{À noter que cela concerne également les projets personnels de grande envergure puisque le développeur à un instant précis n'a pas une vision d'ensemble directement dans sa tête. Il lui a fallu découper cette vision sur des supports écrits et rester méthodique pour ne pas s'embourber dans le déluge de fonctionnalités et de caractéristiques du projet. On peut penser à l'écriture d'une documentation personnelle ou d'un micro-wiki dont la forme peut être quelconque comme des brouillons rassemblés décrivant le fonctionnement de certaines parties du système.} Cette quatrième partie se veut apporter quelques outils pour conserver une stabilité dans le développement d'applications complexes, et apprendre à programmer dans un cadre véritablement concret. Elle vise en particulier :
\begin{itemize}
	\item la recherche autonome de ressources d'apprentissage informatique
	\item le développement "propre" de projets par :
	\begin{itemize}
		\item la minimisation des erreurs incontrôlées \textit{via} la gestion des erreurs et les tests unitaires
		\item l'internationalisation du code\footnote{déjà évoqué en fin de première partie et début de deuxième}
		\item l'utilisation efficiente des commentaires
		\item l'écriture de documentation
	\end{itemize}
	\item le minimum des outils pour la gestion d'un projet informatique :
		\begin{itemize}
			\item \textit{Git}
			\item \textit{Markdown} et les \textit{README}s
			\item un point juridique sur les licences informatiques
		\end{itemize}
\end{itemize}
L'objectif est de rendre le lecteur \textit{autonome} en apportant une conscience de la réalité du monde informatique pour le développement et la programmation.

\hrulefill
\newpage
\end{document}