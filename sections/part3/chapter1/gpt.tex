\documentclass[../../../main.tex]{subfiles}
\graphicspath{{sections/part3/images/}}
\begin{document}
\textbf{Avis de l'auteur quant à l'utilisation d'intelligences artificielles de type LLM (\textit{\underline{L}arge \underline{L}anguage \underline{M}odels} en anglais)}\footnote{Personne ne me l'a demandé, mais bon\dots je vais probablement passer pour un moralisateur invétéré mais merde} :
 
La programmation à l'aide de modèles de langage n'est utile que pour gagner du temps lors de l'écriture de code redondant ou pour l'écriture de documentation (processus en général très peu formateur, redondant et ennuyeux). En effet, l'écriture de ces textes ne demande aucune intelligence et n'apprend rien puisque tout est réfléchi par une intelligence tierce. Cela ne peut être remis en cause à mon avis.
 
Par voie de conséquence, il devient évident que l'utilisation de LLMs pour apprendre à programmer (j'entends par là : écrire à sa place le code\footnote{Et pas demander une explication}) est non seulement inutile mais contre-productif à toute apprentissage, puisque la maîtrise de notions complexes nécessite la maîtrise des fondamentaux, et que rien ne vaut pour véritablement comprendre et assimiler une chose que de la prendre entre quatre yeux et y passer le temps qu'il faut, malgré toute la fatigue que cela peut induire. Dans le cas de projets personnels, il doit être plus épanouissant de réussir à résoudre un problème par soi-même que de copier la solution d'un autre, et c'est de cette façon que non content d'une progression pérenne, le travailleur peut voir en son travail une source de bonheur.\footnote{Aucune influence ici d'un certain programme littéraire de classe préparatoire, c'est évident `,-D}
 
Quant à l'utilité de cet outil pour des problèmes :
\begin{enumerate}
\item dont la résolution sera \textit{évaluée} de quelque manière que ce soit par une partie du corps enseignant
\item devant être fait dans le cadre d'un travail professionnel dans un temps très court et dans un pur but de productivité
\item pour toute autre raison pour laquelle l'apprentissage et la réflexion ne sont pas exigés, et sur un sujet pour lequel le développeur n'éprouve strictement aucun intérêt
\end{enumerate}
, l'auteur :
\begin{enumerate}
\item jugeant le système construit autour de la notation particulièrement désuet et dépourvu d'attraits pédagogiques\footnote{Si tant est que l'on puisse être ``pédagogique'' à forcer un élève dans un enseignement sans jamais tenter de l'intéresser autrement que par le désir d'un nombre plus ou moins élevé qui ne reflétera que très peu son intelligence ou son investissement. Enfin quoi ! Il doit y avoir d'autres moyens d'accrocher les étudiants aux cours, non ?}
\item se trouvant être marxiste\footnote{plutôt marxien pour ceux qui apprécient les nuances}
\item n'appréciant pas de se faire chier pour \textit{aucune} raison
\end{enumerate}
ne voit aucun mal à ChatGPT pour ce qui est de répondre aux exigences d'entités soit incompétentes\footnote{Aucun rapport avec certains charlatans dont les existences ont conduit à l'écriture de ce document. Évidemment\dots} soit n'ayant aucune considération pour l'étudiant/le travailleur.
\end{document}