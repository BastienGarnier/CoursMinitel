\providecommand{\topdir}{../..}
\documentclass[../../main.tex]{subfiles}
\begin{document}
Les deux premières parties de ce livre se sont concentrés sur des aspects assez pratique. Toutefois, on peut observer que certaines notions ont été définies assez rapidement et avec peu de précision, comme par exemple la notion d'\textit{algorithme}. Par ailleurs, certains exercices comme l'\refexercise{Liste des nombres premiers} pointent du doigt l'importance d'un certain aspect théorique mathématique, en particulier en ce qui concerne :
\begin{itemize}
	\item la vitesse d'exécution des algorithmes selon la taille de l'entrée\footnote{Dans cet exemple, deux types de vitesse d'exécution sont observées dans la correction : la vitesse en moyenne et la vitesse dans le pire des cas. On y reviendra dans la suite.}
	\item la démonstration du comportement correct d'un algorithme
\end{itemize}
Le premier point a un intérêt pour des raisons purement pratico-pratique : on veut que les programmes informatiques s'exécutent le plus vite possible dans un contexte donné. Le second point a plus à voir avec certaines applications très précises, notamment pour ce qui est des programmes informatiques dits \og à risques \fg comme les programmes informatiques présents sur les ordinateurs de bord d'avions. On peut aussi penser à la démonstration d'algorithmes pour être certain qu'un circuit microélectronique possède exactement le comportement souhaité\footnote{Dans les faits, on s'appuie pour cela d'automates temporels de propositions logiques, ce qui dépasse très largement le cadre de ce livre. C'était pour illustrer\dots}.

D'autres exercices comme l'\refexercise{Recherche dichotomique} et l'\refexercise{Listes chainées} pointent du doigt le conditionnement de l'efficacité de nos algorithmes selon la manière dont on stocke les données. Dans le cadre d'une formation d'ingénieur, c'est ce point qui sera à retenir, puisque ses implications pratiques sont directes.

Cette partie va donc s'intéresser à présenter :
\begin{itemize}
	\item pour la culture, le modèle de calcul des machines de Turing et la notion de calculabilité qui explore les limites de ce qui peut être calculé par un ordinateur
	\item la notion, très importante dans la pratique, de complexité, qui fournit un cadre théorique pour analyser les performances d'algorithmes
	\item les structures de données élémentaires qui conditionnent l'efficacité des algorithmes
	\item les approches algorithmiques fondamentales pour la résolution de problèmes
\end{itemize}
Les exercices seront une opportunité pour découvrir les algorithmes classiques sur les structures de données élémentaires et les approches algorithmiques.

Dans le cadre de l'ISMIN, cette partie devrait suffire à apporter au lecteur toutes les connaissances théoriques fondamentales exigées durant le cursus\footnote{Voire parfois plus.} en ce qui concerne la programmation.

Il faut avoir conscience que tout ceci n'est qu'une \textit{mise en bouche} qui présente les bases les plus basiques de l'informatique fondamentale/théorique.

\textbf{Lectures d'approfondissement :}
\begin{itemize}
	\item Le cours d'informatique fondamentale de l'école Polytechnique \cite{XFI}, tout en restant introductif, couvre plus largement les notions autour des modèles de calcul
	\item On pourra lire un approfondissement à propos de la notation de Landau \cite{ONotation} accessible sur \textit{ArXive}
	\item La lecture du livre \textit{Éléments d'algorithmique} \cite{EltAlgorithmie}, qui couvre largement le programme d'algorithmique de licence d'informatique, fournira une vision bien plus complète des bases de l'algorithmie
\end{itemize}
Ces livres proposent eux-mêmes d'autres lectures complémentaires.
\end{document}