\providecommand{\topdir}{../..}
\documentclass[../../main.tex]{subfiles}
\begin{document}
Ce document vise à donner une base solide en informatique \og pratique \fg{}, c'est-à-dire en programmation.
Cette pratique est toutefois fondée sur la théorie. Il est donc nécessaire à l'ingénieur de maîtriser
également quelques bases culturelles relatives à l'informatique, et plus précisement à la programmation.

C'est dans cette optique que cette première partie s'attache à décrire certains fondamentaux qui
peuvent, dépendamment de la filière, ne pas avoir été étudiés. En particulier :
\begin{itemize}
	\item mathématiques très utilisées en informatique :
	\begin{itemize}
		\item relations binaires et ensembles ordonnés
		\item induction structurelle
		% \item \textit{rudiments} d'algèbre générale\footnote{Parce-que j'ai pas le niveau pour faire plus et que ça servira pas \textit{anyway} :$|$}
	\end{itemize}
	\item la représentation des données par un alphabet binaire :
	\begin{itemize}
		\item les opérations logiques élémentaires sur les mots binaires
		\item les interprétations des mots binaires pour représenter des nombres entiers ou à virgules
		\item les opérations arithmétiques qui en découlent
		\item l'interprétation des mots binaires pour représenter des caractères textuels
	\end{itemize}
	\item l'architecture fondamentale d'un ordinateur commun
	\begin{itemize}
		\item architecture matérielle naïve de Von Neumann
		\item architecture matérielle plus moderne
		\item architecture logicielle commune
	\end{itemize}
	\item une première approche de la programmation
	\begin{itemize}
		\item introduction (très) élémentaire à l'algorithmique
		\item installation des outils de base nécessaires à la programmation en langage C (Linux/Windows)
		\item compilation et analyse d'un premier programme en langage C
	\end{itemize}
\end{itemize}
Les mathématiques communes à toutes les filières de classes préparatoires sont supposées acquises.

À la fin de cette partie, le lecteur possédera la culture minimale pour apprendre à programmer en comprenant ce qui est manipulé matériellement et logiquement.

\textbf{NOTES :}
\begin{itemize}
	\item La lecture du chapitre 1 est recommandée pour la lecture de la partie 3 (et peut être esquivée sinon)
	\item La lecture des chapitres 2, 3 et 4 est hautement recommandée pour la lecture de la partie 2.
\end{itemize}
\hrulefill
\newpage
\end{document}