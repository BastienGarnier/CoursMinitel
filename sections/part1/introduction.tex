\providecommand{\topdir}{../..}
\documentclass[../../main.tex]{subfiles}
\graphicspath{{sections/part1/images/}}
\begin{document}
Il est un peu provocateur de donner une introduction aux prolégomènes, mais ceux-ci étant particulièrement long cela ne semblait pas tout à fait absurde.

Ce document vise à donner une base solide en informatique \og pratique \fg{}, c'est-à-dire en programmation.
Cette pratique est toutefois fondée sur la théorie. Il est donc nécessaire à l'ingénieur de maîtriser
également quelques bases théoriques relatives à l'informatique, et plus précisement à la programmation.

C'est dans cette optique que cette première partie s'attache à décrire certains fondamentaux qui
peuvent, dépendamment de la filière, ne pas avoir été étudiés. En particulier :
\begin{itemize}
	\item les composants fondamentaux d'un ordinateur classique
	\item les opérations logiques élémentaires sur les mots binaires
	\item les représentations binaires standards des nombres entiers et flottants\footnote{Faut pas haïr les maths, ou ça risque d'être \textit{complicado}}
	\item une première approche simple\footnote{Qui se veut simple en tout cas\dots} de l'idée d'algorithme
	\item quelques premières définitions de vocabulaire technique relatif à l'informatique pratique
\end{itemize}
Pour cela, il est supposé que le/la lecteur/rice a effectué deux ou trois ans de classes préparatoires et
se trouve à la lecture de ce document en entrée d'école d'ingénieur. Un certain bagage mathématique
est donc considéré comme acquis.

Pour finir, un encadrement sera proposé pour installer les outils de base nécessaires à la programmation
en langage C.

\hrulefill
\newpage
\end{document}