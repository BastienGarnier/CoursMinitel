\providecommand{\topdir}{../..}
\documentclass[../../main.tex]{subfiles}
\begin{document}
Ce document vise à donner une base solide en informatique \og pratique \fg{}, c'est-à-dire en programmation.
Cette pratique est toutefois fondée sur la théorie. Il est donc nécessaire à l'ingénieur de maîtriser
également quelques bases culturelles relatives à l'informatique, et plus précisement à la programmation.

C'est dans cette optique que cette première partie s'attache à décrire certains fondamentaux qui
peuvent, dépendamment de la filière, ne pas avoir été étudiés. En particulier :
\begin{itemize}
	\item les composants fondamentaux d'un ordinateur classique
	\item les opérations logiques élémentaires sur les mots binaires
	\item les interpréations des mots binaires pour représenter des nombres entiers, des nombres à virgules et des caractères textuels
	\item une première approche simple de l'idée d'algorithme
	\item quelques premières définitions de vocabulaire technique relatif à l'informatique pratique
	\item l'installation des outils de base nécessaires à la programmation en langage C (Linux/Windows)
\end{itemize}
Pour cela, il est supposé que le/la lecteur/rice a effectué deux ou trois ans de classes préparatoires et
se trouve à la lecture de ce document en entrée d'école d'ingénieur. Un certain bagage mathématique, considéré comme acquis, sera en effet nécessaire. Car ces bases "culturelles" comportent un peu de théorie\footnote{Notamment pour calculer les valeurs numériques associés à des mots binaires}.

À la fin de cette partie, le/la lecteur/rice possédera la culture minimale pour apprendre à programmer en comprenant ce qui est manipulé matériellement et logiquement, et aura une idée de ce que signifie programmer.

\hrulefill
\newpage
\end{document}