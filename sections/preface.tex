\documentclass[../main.tex]{subfiles}
\begin{document}
Ce document a été écrit dans le but premier d'offrir aux étudiants de l'École Nationale Supérieure des
Mines de St-Étienne en cursus ISMIN un support de cours solide pour la programmation en C. De fil
en aiguille, il a finit par couvrir une part plus large de l'informatique qui se trouvait nécessaire à la
bonne compréhension de la programmation et ne pouvait être considérée comme acquise par toutes les
filières de CPGE.

Une bibliographie est disponible en \textbf{Annexe} qui propose de manière thématique certains ouvrages spécialisés\footnote{Je n'ai pas tout lu\dots Pas que l'envie m'en manque, mais classes préparatoires et ISMIN obligent, il n'y a pas toujours le temps de se marrer autant qu'on le voudrait ;(}. Cette bibliographie couvre les thèmes suivants :
\begin{itemize}
	\item Représentation binaire des nombres \cite{MullerEtAl2018}\cite{BitHacks}
	\item Programmation générale en C \cite{KR}\cite{c11}
	\item Programmation sous Linux \cite{LPI}
	\item Compilation \cite{gnumake}\cite{LL}\cite{AL}\cite{Aho}
	\item Programmation système \cite{CSaPP}
	\item Informatique théorique \cite{XFI}
\end{itemize}

Toutes coquilles relevées\footnote{MINITEL n'est pas assez riche pour récompenser en dollars hexadécimales ces braves efforts, mais ce serait gentil malgré tout de bien vouloir nous faire part de ces erreurs ;)}, suggestions d'amélioration ou autres remarques quelconques peuvent être envoyées à la section ``Formation'' du serveur Discord de l'association Minitel de l'ISMIN.

\hrulefill
\newpage
\end{document}