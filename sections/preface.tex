\documentclass[../main.tex]{subfiles}
\begin{document}
\addcontentsline{toc}{section}{Préface}
\subsection*{Section à sauter sans lire}
\addcontentsline{toc}{subsection}{Paragraphe à sauter sans lire}
Merci à Basile Tonlorenzi pour la relecture, les idées et tout et tout ! $\wedge\wedge$

\textbf{\textit{Disclaimer} de genre :} il n'y a pas de neutre en français malheureusement. Je considère que les genres quand ils désignent un groupe d'individus quelconque ou un individu indéterminé, non identifié comme personne, sont sémantiquement nuls et non avenants et donc purs produits du hasard grammatical (exemple : \textit{table} et \textit{tabouret}). La phrase ``Bonjour à tous.'' porte la signification : ``Bonjour à tous les individus présents.''. Le masculin est conservé pour traiter les entités pensées neutres dans le texte, parce-que ça fait moins de lettres à écrire et un accord de moins auquel penser quand j'écris. L'idée est aussi d'alléger l'écriture. Il s'agit des seuls critères dont je puisse faire preuve qui restent objectifs je pense. Certains argueront que je perds autant voire plus de caractères à écrire ce petit paragraphe, mais ça c'est juste pour éviter les dramas quand j'utilise le mot ``lecteur'' à tout bout de champs (entre autre). 

Toutefois, si l'utilisation grammaticale du masculin en permanence rend la lecture de ce livre par certaines lectrices moins aisée, dites et je réécrirais les parties concernées. Je rajoute ce mini-paragraphe là après m'être rendu compte lors de la lecture d'un rapport de jury pour un l'agrégation d'informatique dont une partie était entièrement écrite au féminin que cela m'était moins facile que pour la lecture de la première partie (probablement par manque d'habitude, mais les neurosciences ne sont pas mon fort, donc dans le doute où je me tromperais\dots)

Enfin bref\dots j'ai prévenu qu'il fallait sauter cette section.
\subsection*{Motivation}
\addcontentsline{toc}{subsection}{Motivation}
Ce livre a été écrit dans le but premier d'offrir aux étudiants de l'École Nationale Supérieure des
Mines de St-Étienne en cursus ISMIN un support de cours solide pour la programmation en C. De fil
en aiguille, il a finit par couvrir une part plus large de l'informatique qui se trouvait nécessaire à la
bonne compréhension de la programmation et ne pouvait être considérée comme acquise par toutes les
filières de CPGE.

Certains demanderont -- avec raison -- pourquoi écrire un nouveau livre de programmation quand tant d'autres existent déjà. La première raison, et la principale, est celle de la langue. Exiger la connaissance de la langue anglaise pour apprendre l'informatique ou la programmation constitue un frein sévère pour une grande quantité d'individus. Au lieu de dépenser son énergie à la compréhension, l'assimilation et la maîtrise de connaissances parfois ardûes, on se fatigue et trébuche sur la langue dans laquelle sont exprimées ces connaissances. Il est évident que la maîtrise de l'anglais est nécessaire dans l'industrie informatique, mais tout un chacun devrait pouvoir apprendre si il le veut sans devoir passer quelques années avant à travailler une langue étrangère qui ne constitue que la forme sous laquelle est exprimé le savoir. La seconde raison tient au fait que peu de livres abordent avec précision la programmation en C en considérant que l'étudiant lecteur a un niveau de sortie de classes préparatoires. En effet, dans un premier cas les livres vont considérer une absence totale de connaissances même en mathématique, ce qui amène inévitablement à un manque de rigueur dérangeant, car la programmation est fondamentalement liée à des concepts mathématiques ou tout du moins formelles qui sont parfois esquivés pour des raisons de facilité. Dans un deuxième cas ces livres peuvent présupposer un savoir préalable de nombre de concepts informatiques dont de nombreux étudiants de classes préparatoires n'ont pas connaissances. Par ailleurs, de nombreux livres qui apportent une réflexion théorique, plus orienté informatique que programmation, partent du postulat que le lecteur est déjà à l'aise avec la programmation. L'idée de ce livre est donc d'apporter un alliage de connaissances théoriques et pratiques orientées vers la programmation qui ne réintroduise pas l'intégralité des mathématiques de classes préparatoires, tout en conservant une certaine rigueur.

Toutefois, la légitimité d'un tel cours peut se poser en école d'ingénieurs pour d'autres raisons. Ainsi, les études de la primaire à la classe préparatoire ont habitué les étudiants à ingurgiter de grandes quantités de connaissances pour ainsi dire ``à la petite cuillère''. Or, un ingénieur, en particulier dans le domaine de l'informatique où les technologies évoluent très rapidement, se doit d'être capable d'apprendre \textit{par soi-même}, c'est-à-dire être capable d'aller chercher par soi-même les ressources, bibliographies ou webographiques, nécessaires à son apprentissage. Beaucoup d'enseignants en école d'ingénieurs utilisent ce fait comme argument pour bâcler leur cours. \textit{A contrario}, ce cours a été pensé pour pouvoir être utilisé à la fois comme \textit{catalogue} sur la programmation en langage C, qui décrit rigoureusement les différents points du langage, mais aussi comme une introduction à des connaissances couvrant un spectre plus large de l'informatique et de la programmation. L'idée est de permettre à terme l'autonomie du lecteur pour approfondir de manière technique des sujets spécialisés en fournissant les outils de base nécessaire à leur compréhension.

Un objectif est aussi d'apporter une préparation pour les examens de programmation en C de l'ISMIN, avec une certaine quantité d'exercices corrigés. Les chapitres \textit{Programmer un ordinateur}, \textit{Les fondamentaux du langage} et \textit{Les bases du langage} sont suffisants pour cela. Il n'est pas utile à ceux qui ne souhaitent que valider l'UP d'aller plus loin si tel n'est pas leur souhait.

Pour ceux sûrs de savoir déjà programmer en C, voir le deuxième programme de la section \ref{sec:tricks_r_cr_atifs_ft_basile_} pour le vérifier\footnote{Enfin, juste histoire de mettre l'égo de côté quoi\dots}.

% Il est étrangement faux de considérer que savoir programmer est équivalent à savoir la grammaire de nombreux langages de programmation, à savoir une -- relativement -- grande quantitée de structures de données et d'algorithmes, à savoir comment fonctionne précisement un ordinateur. Ainsi, Alexander Stepanov\footnote{Celui qui a écrit la bibliothèque standard du C++, c'est-à-dire pas la moitié d'un imbécile comme aurait dit mon professeur d'informatique de classes préparatoires.} dit, à propos de cette noble discipline :
% \begin{center}
% \textit{At Stanford there’s one guy who knows, but he’s an emeritus (Donald Knuth).}
% \end{center}

\hrulefill

Une bibliographie est disponible en \textbf{Annexe} qui propose de manière thématique certains ouvrages spécialisés\footnote{Je n'ai pas encore tout lu en entier\dots Pas que l'envie m'en manque, mais classes préparatoires et ISMIN obligent, il n'y a pas toujours le temps de se marrer autant qu'on le voudrait ;(}. Cette bibliographie couvre les thèmes suivants :
\begin{itemize}
	\item Représentation binaire des nombres \cite{MullerEtAl2018}\cite{BitHacks}
	\item Programmation de manière générale \cite{TAOCP}\cite{EoP}\footnote{Pas très spécialisé me direz-vous, mais \textit{The Art of Computer Programming} est si extraordinaire que je ne peux qu'en recommander vivement la lecture.}
	\item Programmation en C \cite{KR}\cite{MSRC}\cite{c11}
	\item Programmation sous Linux \cite{LPI}
	\item Compilation \cite{gnumake}\cite{LL}\cite{AL}\cite{Aho}
	\item Programmation système \cite{CSaPP}
	\item Informatique théorique \cite{XFI}
\end{itemize}

Toutes coquilles relevées\footnote{MINITEL n'est pas assez riche pour récompenser en dollars hexadécimales ces braves efforts, mais ce serait gentil malgré tout de bien vouloir nous faire part des erreurs techniques ;)}, suggestions d'amélioration ou autres remarques quelconques peuvent être envoyées à la section ``Formation'' du serveur Discord de l'association Minitel de l'ISMIN.

\begin{minitelbasicbox}{\textbf{Premier petit apparté} : parenté}
Originellement, ce livre devait, du moins dans mon esprit, être entièrement affilié à l'association étudiante d'informatique de l'école, Minitel. Je considérais, et considère encore, qu'il est mal placé de se mettre en avant dans le cadre d'un travail associatif, puisque l'on travaille pour les autres avant toute chose, pour soi en dernier.

Pourtant, c'est la conscience assez aigüe de mon ignorance de nombreuses choses en matière de programmation et d'informatique qui a changé radicalement l'application de ce point de vue vis-à-vis de ce livre. Cela provient de la certitude que mon travail pouvait être entaché de bévues d'importances assez grandes. Alors, à ne signer que du nom de l'association, les erreurs lui seraient attribuées. Certains vont alléguer que la relecture permet la correction de ces erreurs. Cela est vrai seulement dans le cas d'erreurs locales, purement techniques. Certaines erreurs sont d'un ordre plus large, et n'apparaissent pas clairement. Elles fondent la vision que l'on peut avoir de concepts généraux -- comme l'informatique, la programmation et plus particulièrement la programmation en C. En fait, ces \textit{erreurs} dont j'ai pu entrevoir les possibles existences par la lecture de certaines \oe{}uvres extraordinaires \cite{EoP}\cite{TAOCP} ont parfois amenés à la réécriture -- encore très imparfaite -- d'immenses pans de ce livre. C'est parce-que je ne tiens pas à infliger la responsabilité d'autres inévitables erreurs à l'association Minitel que mon nom est écrit sur la couverture.

Par ailleurs, la qualité stylistique comme pédagogique varie d'une section à une autre, que ce soit dû à la fatigue ou au fait que certaines sections écrites d'abord souffrent d'un manque d'expérience dans la rédaction.
\end{minitelbasicbox}

\hrulefill

\subsection*{Difficulté des exercices}
\addcontentsline{toc}{subsection}{Difficulté des exercices}

Un certain nombre d'exercices sont présents en fin de section, ces sections étant séparés par thème. Ces exercices ont évidemment comme premier objectif de fournir au lecteur des occasions de pratiquer. Un second est de montrer des cas d'application pratique des notions vues dans un chapitre et parfois découvrir de nouvelles notions intéressantes en lien. Ces exercices ont été choisis dans le but d'être intéressants\footnote{Pas toujours réussi hein !}, avec l'objectif ultime qu'en abordant un exercice, le lecteur se dise ``Ah ? Tiens donc, rigolo.''.

On utilise l'échelle de mesure de difficulté utilisé par Donald Knuth\footnote{Le G.O.A.T !}\cite{TAOCP} : 
\begin{center}
	\begin{tabular}{cp{0.8\textwidth}}
	& \textit{Interprétation} \\
	$00$ & Un exercice extrêmement facile qui peut être résolu immédiatement, de tête, si le cours est compris \\
	$10$ & Un problème très simple pour travailler la compréhension du texte. Peut prendre au maximum une minute ou deux et un stylo (ah ! faut du papier aussi !)\\
	$20$ & Un problème moyen pour tester sa compréhension du texte. Peut prendre quinze à vingt minutes pour être résolu complètement \\
	$30$ & Un problème plus difficile, un peu complexe, qui peut prendre plusieurs heures pour être résolu de manière satisfaisante \\
	$40$ & Problème vraiment difficile ou long. Un étudiant doit pouvoir être capable de résoudre le problème en un temps "raisonnable", mais la solution n'est pas trivial \\
	$50$ & Un problème de recherche non encore résolu de manière satisfaisante, bien que beaucoup aient essayés.
	\end{tabular}
\end{center}
\textbf{Remarque : } La difficulté est interpolée \og logarithmiquement \fg, c'est-à-dire qu'un exercice de difficulté $17$ est un peu plus simple qu'un exercice de difficulté $20$, et passablement plus difficile qu'un exercice de difficulté $10$.

\textbf{Sigles spécifiques :} On ajoutera certains symboles à côté de la difficulté pour indiquer des détails sur le type d'exercice :
\begin{itemize}
	\item M : l'exercice implique plus de concepts mathématiques qu'il n'est nécessaire pour un lecteur intéressé uniquement par la programmation
	\item HM : l'exercice implique l'utilisation d'outils mathématiques assez développés qui ne sont pas détaillés dans ce livre
	\item $\bullet$ : l'exercice est particulièrement instructif et utile
\end{itemize}
\textbf{Remarque :} un exercice noté $HM$ n'induit pas \textit{nécessairement} de difficultés.

Ce livre n'a pas comme objectif d'amener à exposer des théories complexes, et l'auteur n'est pas un ressortissant de l'ENS Ulm, donc il n'y aura pas de problèmes côtés à $45$ ou plus.

Par ailleurs, la difficulté d'un exercice est très subjective, puisque ce qui semble simple et intuitif pour l'un peut être un enfer pour l'autre. La difficulté est donc assez approximative, histoire de donner une idée, et que le lecteur ne parte pas bille en tête dans un exercice particulièrement difficile sans se poser de questions ou se complique la vie inutilement sur un exercice \textit{plutôt} simple\footnote{Je n'aimerais pas me chiffonner ou vexer quiconque qui pourrait galérer sur un exercice indiqué "simple"\dots Ça peut aussi être une erreur d'estimation, OU PIRE une faute frappe (genre $13$ au lieu de $31$), faut me prévenir si tel est le cas :)}.

Pour ceux qui voudraient pratiquer plus la programmation\footnote{Pour s'entraîner pour les examens de l'ISMIN par exemple}, la base d'exercices de \textit{Leetcode} est assez fournie (\url{https://leetcode.com/problemset/}). Elle nécessite pour beaucoup exercices les notions présentés dans les trois premières parties.
\newpage
\end{document}