\documentclass[../main.tex]{subfiles}
\begin{document}
\addcontentsline{toc}{section}{Préface}
Adresse de contact valable \textit{a minima} jusqu'à la date sur la couverture : \textit{bastiengarnier17@gmail.com}
\subsection*{Section à sauter sans lire}
\addcontentsline{toc}{subsection}{Paragraphe à sauter sans lire}
Merci à Basile Tonlorenzi pour la relecture, les idées et tout et tout ! $\wedge\wedge$ Content d'avoir un cours relativement complet, après les quelques 10 mois de gestation de ce projet Minitel.

\textbf{À propos du neutre :} il n'y a pas de neutre en français malheureusement. Je considère que les genres quand ils désignent un groupe d'individus quelconque ou un individu indéterminé, non identifié comme personne, sont sémantiquement nuls et non avenants et donc purs produits du hasard grammatical (exemple : \textit{table} et \textit{tabouret} respectivement ``féminin'' et ``masculin''). On pourrait tout aussi bien nommer le masculin et le féminin grammaticaux les formes grammaticales $A$ et $B$ sans que cela ne change rien (c'est-à-dire qu'une table serait à la forme $B$ et un tabouret à la forme $A$, ou inversement selon la convention choisie\dots).

J'utiliserais le masculin grammaticale dans la suite comme forme neutre, par habitude puisqu'il s'agit du standard grammatical pour le neutre depuis quelques centaines d'années. Ainsi, la phrase ``Bonjour à tous.'' porte la signification : ``Bonjour à tous les individus présents.''. Le masculin est ici purement grammaticale et est conservé pour traiter les entités pensées neutres dans le texte. L'idée est aussi d'alléger l'écriture. Si l'utilisation systématique du masculin grammaticale comme neutre pour désigner des entités supposées implicitement humaines dérange, prière d'envoyer un mail avec argumentation. Je réécrirais en conséquence les parties concernées en précisant les deux formes.

Enfin bref\dots j'ai prévenu qu'il fallait sauter cette section.
\subsection*{Motivation}
\addcontentsline{toc}{subsection}{Motivation}
Ce livre a été écrit dans le but premier d'offrir aux étudiants de l'École Nationale Supérieure des
Mines de St-Étienne en cursus ISMIN un support de cours \textit{solide et complet} pour la programmation en C. De fil
en aiguille, il a finit par couvrir une part plus large de l'informatique qui se trouvait nécessaire à la
bonne compréhension de la programmation et ne pouvait être considérée comme acquise par toutes les
filières de CPGE.

Certains demanderont -- en toute légitimité -- pourquoi écrire un nouveau livre de programmation quand tant d'autres existent déjà. À cela plusieurs raisons :
\begin{enumerate}
	\item Ce cours assure la localité des informations. Le programme d'informatique de l'ISMIN concerne plusieurs domaines particuliers de l'informatique qui peuvent être très différents, bien que liés. Pour se renseigner plus avant sur les sujets abordés en cours, ou trouver des ressources de révision, il faut alors parcourir de nombreux ouvrages différents. D'abord un étudiant n'a pas nécessairement connaissance des références littéraires sur le sujet, ce qui peut l'amener à devoir itérer ses recherches un certain nombre de fois avant d'arriver sur une source d'information claire et précise qui couvre exactement l'objet de ses désirs. Deuxièmement, ces ouvrages couvrent leur thème souvent de manière très approfondie et un étudiant n'est pas toujours intéressé par l'intégralité du contenu. Ce livre de Minitel veut couvrir les thèmes du cursus en introduisant juste assez de notions pour garantir la rigueur sans pour autant (trop) déborder en hors-sujet vis-à-vis du programme.
	\item Exiger la connaissance de la langue anglaise pour apprendre l'informatique ou la programmation constitue un frein sévère pour une grande quantité d'individus. Au lieu de dépenser son énergie à la compréhension, l'assimilation et la maîtrise de connaissances parfois ardûes, on se fatigue et trébuche sur la langue dans laquelle sont exprimées ces connaissances. Il est évident que la maîtrise de l'anglais est nécessaire dans l'industrie informatique, mais tout un chacun devrait pouvoir apprendre si il le veut sans devoir passer quelques années avant à travailler une langue étrangère qui ne constitue que la forme sous laquelle est exprimé le savoir.
	\item Peu de livres abordent avec précision la programmation en C en considérant que l'étudiant lecteur a un niveau de sortie de classes préparatoires. En effet, dans un premier cas les livres vont considérer une absence totale de connaissances même en mathématique, ce qui amène inévitablement à un manque de rigueur dérangeant, car la programmation est fondamentalement liée à des concepts mathématiques qui sont parfois esquivés pour des raisons de facilité. Dans un deuxième cas ces livres peuvent présupposer un savoir préalable de quantité de concepts informatiques dont de nombreux étudiants de classes préparatoires n'ont pas connaissances. Par ailleurs, de nombreux livres qui apportent une réflexion théorique, plus orienté informatique que programmation, partent du postulat que le lecteur est déjà à l'aise avec la programmation. L'idée de ce livre est donc d'apporter un alliage de connaissances théoriques et pratiques orientées vers la programmation (impérative) qui ne réintroduise pas des mathématiques de classes préparatoires, tout en conservant une certaine rigueur.
	\item Ce livre est éditable par l'association Minitel et peut donc être étendu aux ajouts de notions résultant de réformes du programme de l'ISMIN. De plus, si certaines explications ne sont pas claires ou qu'un point nécessite des exemples et explications supplémentaires, ce peut être ajouté. La mise en page aussi peut être retravaillée. De manière générale, la possibilité pour les étudiants d'éditer le cours selon leurs besoins le rend particulièrement flexible. Le défaut est évidemment qu'il n'est pas certain que tout contenu ajouté soit parfaitement valide et sans fautes. C'est pour cela qu'il faut s'attacher à la plus grande rigueur, par référencement bibliographique des notions, démonstration des propositions et de manière générale, argumentation de toute assertion. La présente édition n'est pas parfaite. Il est espéré qu'elle soit complétée/améliorée avec le temps.
\end{enumerate}
Toutefois, la légitimité d'un tel cours peut se poser en école d'ingénieurs pour d'autres raisons. Ainsi, les études de la primaire à la classe préparatoire ont habitué les étudiants à ingurgiter de grandes quantités de connaissances pour ainsi dire ``à la petite cuillère''. Or, un ingénieur, en particulier dans le domaine de l'informatique où les technologies évoluent très rapidement, se doit d'être capable d'apprendre \textit{par soi-même}, c'est-à-dire être capable d'aller chercher par soi-même les ressources, bibliographiques ou webographiques, nécessaires à son apprentissage. Beaucoup d'enseignants en école d'ingénieurs utilisent ce fait comme argument pour bâcler leur cours. \textit{A contrario}, ce cours a été pensé pour pouvoir être utilisé à la fois comme \textit{catalogue} sur la programmation en langage C, qui décrit rigoureusement les différents points du langage, mais aussi comme une introduction à des connaissances couvrant un spectre plus large de l'informatique et de la programmation. L'idée est de permettre à terme l'autonomie du lecteur pour approfondir de manière technique des sujets spécialisés en fournissant les outils de base nécessaire à leur compréhension.

Ainsi, une bibliographie est disponible en \textbf{Annexe} qui propose de manière thématique certains ouvrages spécialisés de référence\footnote{Je n'ai pas encore tout lu en entier\dots Pas que l'envie m'en manque, mais classes préparatoires et ISMIN obligent, il n'y a pas toujours le temps de se marrer autant qu'on le voudrait ;(}. Cette bibliographie couvre (entre autres) les thèmes suivants :
\begin{itemize}
	\item Représentation binaire des nombres \cite{MullerEtAl2018}\cite{BitHacks}
	\item Programmation de manière générale \cite{EoP}
	\item Langage C \cite{KR}\cite{MSRC}\cite{c11}
	\item Programmation système sous Linux \cite{LPI}
	\item Compilation \cite{gnumake}\cite{LL}\cite{AL}\cite{Aho}
	\item Programmation bas-niveau \cite{CSaPP}
	\item Algorithmique \cite{EltAlgorithmie}\cite{TAOCP}
	\item Informatique fondamentale \cite{ONotation}\cite{XFI}
\end{itemize}

L'objectif est d'apporter une préparation pour les examens de programmation en C de l'ISMIN, nommément \textit{Algo/Prog} 1 et 2. Les chapitres \ref{cha:programmer_un_ordinateur}, \ref{chap:fondamentaux_du_langage} et \ref{chap:fondamentaux_du_langage} sont suffisants pour \textit{Algo/Prog 1}. La lecture du chapitre \ref{cha:structures_de_donn_es_classiques} est suffisante pour \textit{Algo/Prog 2}. La lecture du chapitre \ref{chap:outils} est particulièrement recommandée. Le reste sert à la rigueur, à l'exhaustivité et se veut un complément technique d'intérêt.

La programmation en C/C++ et la maîtrise de domaines de l'informatique sont aussi exigées pour d'autres cours :
\begin{itemize}
	\item \textit{Programmation Système} en 1A
	\item \textit{Architecture des processeurs (1 et 2)} en 1A et 2A
	\item \textit{Systèmes à microcontrôleurs (1 et 2)} en 1A et 2A
	\item \textit{Programmation Orientée Objets} en 2A
	\item \textit{Cryptographie} en 2A
	\item \textit{Optimisation combinatoire} en 2A
	\item \textit{Intelligence Artificielle} en 2A
	\item \textit{Sécurité des réseaux informatiques} en 2A
\end{itemize}
Ce livre ne couvre pas \textit{encore} les points théoriques et techniques abordés durant ces cours. À voir\dots (priorité pour Architecture des processeurs, y a pas de support de cours en 2A, et SAM le cours est \dots)

\hrulefill

Toutes coquilles relevées\footnote{MINITEL n'est pas assez riche pour récompenser en dollars hexadécimales ces braves efforts, mais ce serait gentil malgré tout de bien vouloir nous faire part des erreurs techniques ;)}, suggestions d'amélioration ou autres remarques quelconques peuvent être envoyées à la section ``Formation'' du serveur Discord de l'association Minitel de l'ISMIN.

La qualité stylistique comme pédagogique varie d'une section à une autre, que ce soit dû à la fatigue, au fait que certaines sections écrites d'abord souffrent d'un manque d'expérience dans la rédaction et que d'autres sections écrites en dernières souffrent de manquement dans la relecture et l'affinage.

\textbf{À propos de la licence :} Une association n'a pas but à être lucrative. La licence \textsc{CC-BY-NC-SA} spécifie que :
\begin{itemize}
	\item L'oeuvre peut être reproduite, distribuée et modifiée en accord avec les points ci-dessous.
	\item Le nom de chaque auteur de ce document doit être cité (mise-à-jour à chaque nouvelle version)
	\item Ce document ne peut pas être édité ou modifié sous une autre licence\footnote{Il existe une copie de l'auteur originel qui atteste de la licence d'origine. \textit{Just to say\dots}}.
	\item Tout usage commercial de ce document est prohibé.
\end{itemize}

\hrulefill

\subsection*{Difficulté des exercices}
\addcontentsline{toc}{subsection}{Difficulté des exercices}

Un certain nombre d'exercices sont présents en fin de section, ces sections étant séparés par thème. Ces exercices ont évidemment comme premier objectif de fournir au lecteur des occasions de pratiquer. Un second est de montrer des cas d'application pratique des notions vues dans un chapitre et parfois découvrir de nouvelles notions intéressantes en lien. Ces exercices ont été choisis dans le but d'être intéressants\footnote{Pas toujours réussi hein !}, avec l'objectif ultime qu'en abordant un exercice, le lecteur se dise ``Ah ? Tiens donc, rigolo.''\footnote{Encore moins réussi\dots}.

On utilise l'échelle de mesure de difficulté utilisé par Donald Knuth\footnote{Le G.O.A.T !}\cite{TAOCP} : 
\begin{center}
	\begin{tabular}{cp{0.8\textwidth}}
	& \textit{Interprétation} \\
	$00$ & Un exercice extrêmement facile qui peut être résolu immédiatement, de tête, si le cours est compris \\
	$10$ & Un problème très simple pour travailler la compréhension du texte. Peut prendre au maximum une minute ou deux et un stylo (ah ! faut du papier aussi !)\\
	$20$ & Un problème moyen pour tester sa compréhension du texte. Peut prendre quinze à vingt minutes pour être résolu complètement \\
	$30$ & Un problème plus difficile, un peu complexe, qui peut prendre plusieurs heures pour être résolu de manière satisfaisante \\
	$40$ & Problème vraiment difficile ou long. Un étudiant doit pouvoir être capable de résoudre le problème en un temps "raisonnable", mais la solution n'est pas trivial \\
	$50$ & Un problème de recherche non encore résolu de manière satisfaisante, bien que beaucoup aient essayés.
	\end{tabular}
\end{center}
\textbf{Remarque : } La difficulté est interpolée \og logarithmiquement \fg, c'est-à-dire qu'un exercice de difficulté $17$ est un peu plus simple qu'un exercice de difficulté $20$, et passablement plus difficile qu'un exercice de difficulté $10$.

\textbf{Sigles spécifiques :} On ajoutera certains symboles à côté de la difficulté pour indiquer des détails sur le type d'exercice :
\begin{itemize}
	\item M : l'exercice implique plus de concepts mathématiques qu'il n'est nécessaire pour un lecteur intéressé uniquement par la programmation
	\item HM : l'exercice implique l'utilisation d'outils mathématiques assez développés qui ne sont pas détaillés dans ce livre
	\item $\bullet$ : l'exercice est particulièrement instructif et utile
\end{itemize}
\textbf{Remarque :} un exercice noté $HM$ n'induit pas \textit{nécessairement} une difficulté supplémentaire extraordinaire.

Ce livre n'a pas comme objectif d'amener à exposer des théories complexes, et l'auteur\footnote{\textit{Les} auteurs un jour ?} n'est pas un ressortissant de l'ENS Ulm, donc il n'y aura pas de problèmes côtés à $45$ ou plus.

Par ailleurs, la difficulté d'un exercice est très subjective, puisque ce qui semble simple et intuitif pour l'un peut être un enfer pour l'autre. La difficulté est donc assez approximative, histoire de donner une idée, et que le lecteur ne parte pas bille en tête dans un exercice particulièrement difficile sans se poser de questions ou se complique la vie inutilement sur un exercice \textit{plutôt} simple\footnote{Je n'aimerais pas me chiffonner ou vexer quiconque qui pourrait galérer sur un exercice indiqué "simple"\dots Ça peut aussi être une erreur d'estimation, OU PIRE nue ftaue ed fappre (genre $13$ au lieu de $31$), faut me prévenir si tel est le cas :)}.

Pour ceux qui voudraient pratiquer plus la programmation\footnote{Pour s'entraîner pour les examens de l'ISMIN par exemple}, la banque d'exercices de \textit{Leetcode} est assez fournie (\url{https://leetcode.com/problemset/}). Elle nécessite pour beaucoup d'exercices les notions présentées dans les trois premières parties.
\newpage
\end{document}